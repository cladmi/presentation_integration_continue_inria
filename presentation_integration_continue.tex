%  Beamer slide example.

\documentclass[9pt]{beamer}
\usepackage[utf8]{inputenc}
\usetheme{inria}
\usepackage{helvet}
\usepackage{comment}

\author{Maurice Bremond \and Gaëtan Harter}

\title[Intégration continue]{L'Intégration Continue}
% \subtitle{Dans un contexte de développement Inria}
\subtitle{Présentation IJD}



% Automatically insert a "new section" page at each section.
\AtBeginSection[]{
  \begin{frame}[plain]
    \partpage
  \end{frame}
}
% \inriaswitchcolors COLOR
%
% Where COLOR is one of red, blue, orange, darkblue, violet,
% pastelgreen, grey, or green.
\newcommand{\inriaswitchcolors}[1]{%
\pgfaliasimage{figfootline}{figfootline-#1}% !!!
\pgfaliasimage{figbackground}{figbackground-#1}% !!!
\pgfaliasimage{figbackground}{figbackground-#1}% !!!
}
% starting the document
% *********************
\begin{document}
% titlepage
% ---------
\begin{frame}[plain]
\titlepage
\end{frame}
% table of contents
% -----------------
\begin{frame}{\textcolor{inriaGrey}{Table des matières}}
  \tableofcontents
\end{frame}



% Introduction
% ************

\inriaswitchcolors{red}
\section{Quid de l'intégration continue}

\subsection{Qu'est-ce que c'est}
\begin{frame}{Qu'est-ce que c'est}

Contenu

\end{frame}



% L'intégration continue à Inria
% ******************************
\inriaswitchcolors{blue}
\section{L'intégration continue à Inria}




% Retour d'expérience
% *******************
\inriaswitchcolors{green}
\section{Retour d'expérience}

\subsection{Contexte de développement}
\begin{frame}{Contexte de développement}
\end{frame}

\subsection{Outils et procédures mises en places}
\begin{frame}{Outils et procédures mises en places}
\end{frame}

\subsection{Bilan}
\begin{frame}{Bilan}
\end{frame}

\end{document}
